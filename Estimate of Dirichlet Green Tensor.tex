
\documentclass[12pt]{iopart}
\pdfoutput=1
\usepackage{iopams}
\usepackage{amssymb, epsfig}
%\usepackage{amsmath, amssymb,epsfig}
\usepackage{latexsym}

%\usepackage[hypertex,hyperindex]{hyperref}
%\usepackage{showkeys}
\usepackage{graphicx}
\usepackage{color}

\newcommand{\pf}{\mbox{pf}}

\begin{document}

\bibliographystyle{plain}
\def\debproof{\noindent {\bf Proof.} }
\def\finproof{\hfill {\small $\Box$} \\}
%\renewcommand{\theequation}{\arabic{section}.\arabic{equation}}

\makeatletter % `@' now normal "letter"
\@addtoreset{equation}{section}
\makeatother  % `@' is restored as "non-letter"
\renewcommand\theequation{{\thesection}.{\arabic{equation}}}
\title[]{}



\maketitle
\newcommand{\eps}{\varepsilon}
\newcommand{\RR}{\mathcal{R}}
\newtheorem{lem}{Lemma}[section]
\newtheorem{prop}{Proposition}[section]
\newtheorem{cor}{Corollary}[section]
\newtheorem{thm}{Theorem}[section]
\newtheorem{rem}{Remark}[section]
\newtheorem{alg}{Algorithm}[section]
\newtheorem{assum}{Assumption}[section]
\newtheorem{definition}{Definition}[section]


\newcounter{RomanNumber}
\newcommand{\MyRoman}[1]{\rm\setcounter{RomanNumber}{#1}\Roman{RomanNumber}}

\newcommand{\bL}{\mathbf{L}}
\newcommand{\bH}{\mathbf{H}}
\newcommand{\bW}{\mathbf{W}}
\newcommand{\bP}{\mathbf{P}}
\newcommand{\bQ}{\mathbf{Q}}
\newcommand{\bp}{\mathbf{p}}
\newcommand{\bq}{\mathbf{q}}
\newcommand{\uL}{u_{_{\rm L}}}
\newcommand{\vL}{v_{_{\rm L}}}
\newcommand{\tuL}{\tilde u_{_{\rm L}}}
\newcommand{\tvL}{\tilde v_{_{\rm L}}}
\newcommand{\fL}{f_{_{\rm L}}}
\newcommand{\gL}{g_{_{\rm L}}}
\newcommand{\bpL}{\bp_{_{\rm L}}}
\newcommand{\bqL}{\bq_{_{\rm L}}}
\newcommand{\tbpL}{\tilde{\bp}_{_{\rm L}}}
\newcommand{\tbqL}{\tilde{\bq}_{_{\rm L}}}
\newcommand{\tbpLf}{\tilde{\bp}_{_{\rm L,1}}}
\newcommand{\tbpLs}{\tilde{\bp}_{_{\rm L,2}}}
\newcommand{\tbqLf}{\tilde{\bq}_{_{\rm L,1}}}
\newcommand{\tbqLs}{\tilde{\bq}_{_{\rm L,2}}}
\newcommand{\bn}{\nu}
\newcommand{\bv}{\mathbf{v}}
\newcommand{\om}{\omega}
\newcommand{\pa}{\partial}
\newcommand{\la}{\langle}
\newcommand{\ra}{\rangle}
\newcommand{\lla}{\la{\hskip -2pt}\la}
\newcommand{\rra}{\ra{\hskip -2pt}\ra}
\newcommand{\jj}{\|{\hskip -0.8pt} |}
\newcommand{\al}{\alpha}
\newcommand{\ze}{\zeta}
\newcommand{\si}{\sigma}
\newcommand{\ep}{\varepsilon}
\newcommand{\na}{\nabla}
\newcommand{\vp}{\varphi}
\newcommand{\ga}{\gamma}
\newcommand{\Ga}{\Gamma}
\newcommand{\Om}{\Omega}
\newcommand{\de}{\delta}
\newcommand{\Th}{\Theta}
\newcommand{\De}{\Delta}
\newcommand{\Lam}{\Lambda}
\newcommand{\lam}{\lambda}
\newcommand{\tri}{\triangle}
\newcommand{\lj}{[{\hskip -2pt} [}
\newcommand{\rj}{]{\hskip -2pt} ]}
\newcommand{\bks}{\backslash}
%\newcommand{\diag}{\mathrm{diag}}
\newcommand{\diam}{\mathrm{diam}}
\newcommand{\osc}{\mathrm{osc}}
\newcommand{\meas}{\mathrm{meas}}
\newcommand{\dist}{\mathrm{dist}}

\newcommand{\mL}{\mathscr{L}}
\newcommand{\cT}{{\cal T}}
\newcommand{\cM}{{\cal M}}
\newcommand{\cE}{{\cal E}}
\newcommand{\cL}{{\cal L}}
\newcommand{\cF}{{\cal F}}
\newcommand{\cB}{{\cal B}}
\newcommand{\PML}{{\rm PML}}
\newcommand{\FEM}{{\rm FEM}}
\newcommand{\rd}{\,\mathrm{d}}

\renewcommand{\i}{\mathbf{i}}
\renewcommand{\v}{\mathbf{v}}
\renewcommand{\u}{\mathbf{u}}
\renewcommand{\r}{\mathbf{r}}
\newcommand{\R}{{\mathbb{R}}}
\newcommand{\Z}{{\mathbb{Z}}}
\newcommand{\C}{{\mathbb{C}}}
\renewcommand{\Re}{\mathrm{Re}\,}
\renewcommand{\Im}{\mathrm{Im}\,}
\renewcommand{\div}{\mathrm{div}}
\newcommand{\curl}{\mathrm{curl}}
\newcommand{\Curl}{\mathbf{curl}}


%%%%%%%%%%%%%%%%%%%%%%%%%%%%%%%%%%%%%%%%%%%%%%%%%%%%%%%%%%%%%%%%%%%%
\newcommand{\be}{\begin{eqnarray}}
\newcommand{\ee}{\end{eqnarray}}
\newcommand{\ben}{\begin{eqnarray*}}
\newcommand{\een}{\end{eqnarray*}}
\newcommand{\nn}{\nonumber}


\section{Estimate of Dirichlet Green Tensor}
We need the following slight generalization of Van der Corput lemma for the oscillatory integral \cite[P.152]{grafakos}.
\begin{lem}\label{van}
	Let $-\infty<a<b<\infty$, and $u$ is a $C^k$ function $u$ in $(a,b)$. \\
 1. If $|u'(t)|\ge 1$ for $t\in (a,b)$ and $u'$ is monotone in (a,b), then for any $\phi(t)$ in $(a,b)$ with integrable derivatives
	\ben
	\left|\int^b_a e^{\i\lambda u(t)}\phi(t)dt\right|\le
	3\lambda^{-1}\left[|\phi(b)|+\int^b_a|\phi'(t)|dt\right].
	\een
 2. For all $k\geq2$, if $|u^{(k)}(t)|\ge 1$ for $t\in (a,b)$ , then for any $\phi(t)$ in $(a,b)$ with integrable derivatives
	\ben
	\left|\int^b_a e^{\i\lambda u(t)}\phi(t)dt\right|\le
	12k\lambda^{-1/k}\left[|\phi(b)|+\int^b_a|\phi'(t)|dt\right].
	\een
\end{lem}
\debproof
The assertion can be proved by extending the Van der Corptut lemma in \cite{grafakos}. Here we omit the details.
\finproof
We recall following lemma, see e.g. \cite{Wong_Asymptotic}:
\begin{lem} \label{asym_frac}
	Let $F(\lambda,a)=\int_{0}^{a} t^{\alpha-1}f(t)e^{\i\lambda t}dt$ where $0<a\leq+\infty$, $0<\alpha<1$ and $t^{\alpha-1}f\in L^1(0,a)$, then we have
	\be
	|F(\lambda,a)|\leq C(\frac{1}{\lambda^\alpha}f(0)+\frac{1}{\lambda}(a^{\alpha-1}f(a)+|t^{\alpha-1}f|_{L^1(0,a)})
	\ee
\end{lem}
\begin{lem}\label{pr_dgreen}
	Let $f(\xi,\mu_s,\mu_p)=g(\xi,\mu_s,\mu_p)/\gamma(\xi,\mu_s,\mu_p)$ where g(x,y,z) is a homogeneous quadratic polynomial with respect to x,y,z. Let $a,b>0$ and $\rho=\sqrt{a^2+b^2}$. Assume $\kappa=k_p/k_s$ and $k_s\rho >1$, then we have
	\be \label{pr_dgreen1} \nn
	\Big|\int_\R f(\xi,\mu_s,\mu_p) e^{\i(\mu_s b +\xi a)}d\xi
	-f_{\xi=\frac{k_s a}{\rho}}\frac{k_s b}{\rho}(\frac{2\pi}{k_s \rho})^{1/2}e^{\i(k_s\rho-\frac{\pi}{4})}\Big|
	\\ \hspace{-0.5cm}
	 \leq C(\frac{k_s b}{\rho(k_s\rho)^{3/4}}+\frac{k_s a}{\rho(k_s\rho)^{5/4}})
	\ee
	where C is only dependent on $\kappa$.
\end{lem}
\debproof
Let $I(a,b)=\int_\R f(\xi,\mu_s,\mu_p) e^{\i(\mu_s b +\xi a)}d\xi$ . To simplify the integral, the standard substitution $\xi=k_s\sin t$ is made, taking the $\xi$-plane to a strip $-\pi/2<\Re t <\pi/2$ in the t-plane, and the real axis in the $\xi$-plane onto the path L from $-\pi/2+\i\infty\rightarrow-\pi/2\rightarrow\pi/2\rightarrow\pi/2-\i\infty$ in the t-plane. Then $I(a,b):=I(\rho,\phi)$ becomes(Let a=$\rho \sin\phi$  and b=$\rho\cos\phi$, $0<\phi<\pi/2$)
\be
k_s \int_L f(\sin t,\cos t,(\kappa^2-\sin^2 t)^{1/2})\cos t \ e^{\i k_s\rho(\cos (t-\phi))} dt
\ee
Taking the shift transformation of t and using cauchy integral theorem, we can obtain the representation of I(a,b):
\ben \hspace{-1.5cm}
k_s\int_L f(\sin (t+\phi),\cos (t+\phi),(\kappa^2-\sin^2 (t+\phi))^{1/2})\cos (t+\phi) e^{\i k_s\rho(\cos t)} dt \\\hspace{-2cm}
=k_s \cos \phi \int_L f(\sin (t+\phi),\cos (t+\phi),(\kappa^2-\sin^2 (t+\phi))^{1/2})\cos t \ e^{\i k_s\rho(\cos t)} dt \\\hspace{-2cm}
-k_s \sin \phi \int_L f(\sin (t+\phi),\cos (t+\phi),(\kappa^2-\sin^2 (t+\phi))^{1/2})\sin t \ e^{\i k_s\rho(\cos t)} dt \\\hspace{-2cm}
:=k_s (\cos\phi \ I_1+\sin\phi \ I_2)
\een
For $I_2$, using integration by parts on path $L$ first, we have
\be \hspace{-2cm}
I_2=\frac{1}{\i k_s\rho} \int_L f(\sin (t+\phi),\cos (t+\phi),(\kappa^2-\sin^2 (t+\phi))^{1/2}) d \ e^{\i(k_s\rho \cos t)} \\ \hspace{-1.5cm}
=-\frac{1}{\i k_s\rho} \int_{L} \frac{\pa f(\sin (t+\phi),\cos (t+\phi),(\kappa^2-\sin^2 (t+\phi))^{1/2})}{\pa t
} \  e^{\i(k_s\rho \cos t)} dt \\ \hspace{-1.5cm}
:=\frac{1}{ k_s\rho}I_3
\ee
Therefore,
\be \label{d_expan}
I(\rho,\phi)=k_s(\cos\phi I_1 +\frac{\sin\phi}{k_s\rho}I_3)
\ee
First, we define $0<\phi_1<\phi_\kappa<\phi_2<\pi/2$ such that $\sin(\phi_\kappa)=\kappa,\sin(\phi_1)<\kappa/2,\sin(\phi_2)>(1+\kappa)/2$ and $\cos(\phi_1)>(1+\kappa)/2,\cos(\phi_2)<\kappa/2$. Now, we claim that \\
1. If $\phi\in(0,\phi_1)\cup(\phi_2,\pi/2)$, we have
\be \label{pr_dgreen3}
\Big|I_1-f(\sin\phi,\cos\phi,(\kappa^2-\sin^2\phi)^{1/2})(\frac{2\pi}{k_s \rho})^{1/2}e^{\i(k_s\rho-\frac{\pi}{4})}\Big|
\leq C\frac{1}{k_s\rho} \\
|I_3|\leq C\frac{1}{(k_s\rho)^{1/2}}
\ee
where $C$ is independant of $\phi$.\\
2. If $\phi\in[\phi_1,\phi_2]$ and $\phi\neq\phi_\kappa$, we have
\be
\Big|I_1-f(\sin\phi,\cos\phi,(\kappa^2-\sin^2\phi)^{1/2})(\frac{2\pi}{k_s \rho})^{1/2}e^{\i(k_s\rho-\frac{\pi}{4})}\Big|
\leq C(\phi)\frac{1}{k_s\rho} \\
|I_3|\leq C(\phi)\frac{1}{(k_s\rho)^{1/2}}
\ee
3. If $\phi=\phi_\kappa$, we have
\be
\Big|I_1-f(\sin\phi,\cos\phi,0)(\frac{2\pi}{k_s \rho})^{1/2}e^{\i(k_s\rho-\frac{\pi}{4})}\Big|
\leq C\frac{1}{(k_s\rho)^{3/4}} \\
|I_3|\leq C(\phi)\frac{1}{(k_s\rho)^{1/4}}
\ee
Since $I(\rho,\phi)$ is a continuous function, we can obtain estimate (\ref{pr_dgreen1}) soon by the claim and equality (\ref{d_expan}). We now proceed with the proof of the claim above.

1. For the claim 1, wo only give the proof when $\phi\in(\phi_2,\pi/2)$ since the similar proof can be adjusted to the other case. By the convention of $\phi_2$, for any $\phi\in(\phi_2,\pi/2)$, there exists $0<\delta<\pi/4$ only dependent on $\kappa$ such that
\be \label{d_convention1}
|\sin(t+\phi)|>(1+2\kappa)/3,|\cos(t+\phi)|<2\kappa/3
\ee
for any $t\in(-\delta,\delta)$ while \be
|\cos(t+\phi)|>(1+2\kappa)/3, |\sin(t+\phi)|<2\kappa/3
\ee
 for any $t\in(-\pi/2,-\pi/2+\delta)\cup(\pi/2-\delta,\pi/2)$. Let $\chi_\delta\in C^\infty_0(-\pi/2,\pi/2)$ be the cut-off function with that $0\leq\chi_\delta\leq1$, $\chi_\delta=1$ in $(-\delta/2,\delta/2)$ and $\chi_\delta=0$  in $L\bks(-\delta,\delta)$. Then we can divide $I_1$ into two parts such that
\be \nn
I_1=\int_L f(t)\cos t e^{\i k_s\rho\cos t}dt\\ \label{I_split}
=\int_\R f(t)\cos t \chi_\delta(t)e^{\i k_s\rho\cos t} dt +
\int_{L'} f(t)\cos t (1-\chi_\delta(t))e^{\i k_s\rho\cos t} dt  \\ \nn
=: I_{11} + I_{12}
\ee
where $L'=L\bks(-\delta/2,\delta/2)$ and $f(t):=f(\sin (t+\phi),\cos (t+\phi),(\kappa^2-\sin^2 (t+\phi))^{1/2})$. Let $g_\delta(t)=f(t)\cos t \chi_\delta(t)$ and  subtitating $t(s)=2\arcsin s/2$ for t in $I_{11}$, we have
\be
I_{11}=\int_\R g_\delta(t(s))\frac{1}{\sqrt{1-s^2/4}}e^{\i k_s \rho}e^{-\i k_s\rho s^2/2} ds
\ee
Let $h_\delta(s)=g_\delta(t(s))\frac{1}{\sqrt{1-s^2/4}}$. It is easy to see that $h_\delta(s)\in C^\infty_0(-2\sin \delta/2,2\sin\delta/2)$. By the lemma of the stationary phase for quadratic term in \cite{Evans2010}, we have
\be
I_{11}=e^{\i k_s \rho}\int_\R h_\delta(s)e^{-\i\frac{k_s\rho}{2}s^2}ds=
e^{\i k_s \rho}\int_\R \widehat{h_\delta}(y)\alpha(-y)dy
\ee
where
\be
\alpha(y)=(\frac{1}{2\pi k_s \rho})^{1/2}e^{-\i\pi/4}e^{\frac{\i}{2 k_s \rho}y^2} \\
=(\frac{1}{2\pi k_s \rho})^{1/2}e^{-\i\pi/4}(1+O(\frac{y^2}{k_s\rho}))
\ee
Consequently
\be
I_{11}=(\frac{1}{2\pi k_s \rho})^{1/2}e^{\i k_s \rho-\i\pi/4}
\int_\R \widehat{h_\delta}(y)(1+\frac{1}{k_s \rho}O(y^2)) dy
\ee
But $\int_\R \widehat{h_\delta}(y)dy=2\pi h_\delta(0)$ and $|\int_\R \widehat{h_\delta}(y)y^2 dy|<C$ since $|\widehat{h_\delta}(y)|< C_1$ and $|\widehat{h_\delta}(y)|<C_2/y^4$ where $C,C_1,C_2$ is independent of $\phi$.
It turns to estimate $I_{12}$. Using integration by parts, we obtain
\be
|I_{12}|=\Bigg|\frac{1}{ k_s \rho}\int_{L'}(f(t)\cos t (1-\chi_\delta(t)/\sin t)' e^{\i k_s\rho\cos t}  dt\Bigg| \\
\leq \frac{1}{k_s \rho}\Bigg(\int_{L\bks(-\frac{\pi}{2},\frac{\pi}{2})}|(f(t)\cos t (1-\chi_\delta(t)/\sin t)'| e^{\i \cos t} dt\\
+\int_{(-\frac{\pi}{2},\frac{\pi}{2})\bks(-\frac{\delta}{2},\frac{\delta}{2})}|(f(t)\cos t (1-\chi_\delta(t)/\sin t)'|  dt \Bigg) \\
\leq C\frac{1}{k_s \rho}
\ee
Then inequality (\ref{pr_dgreen3}) follows.


For $I_3$, we split the integral path $L$ into $L_1=(-\pi/2,\pi/2)$ and $L_2=(-\pi/2+\i\infty,-\pi/2)\cup(\pi/2,\pi/2-\i\infty)$, then we have corresponding representation: $I_3=I_{31}+I_{32}$.
Then $|I_{32}|\leq C/(k_s\rho)$ can be proved by the same method used above. Following a tedious computation, we obtain a simple form of $\pa f/\pa t$:
\be
\frac{\pa f}{\pa t}&=&\frac{(\gamma\pa_t g  -g \pa_t \gamma)(\kappa^2-\sin^2 t)^{1/2}}{(\sin^2 t +\cos t (\kappa^2-\sin^2 t)^{1/2})^2}\frac{1}{(\kappa^2-\sin^2 t)^{1/2}} \\
&:=&\frac{h(\sin (t+\phi),\cos (t+\phi),(\kappa^2-\sin^2 (t+\phi))^{1/2})}{(\kappa^2-\sin^2 t)^{1/2}}
\ee
where $h$ and $\pa h/\pa t$ are integrable on path $L_1$. Let define $t_1,t_2\in \chi_1=(-\pi/2+\delta,-\delta)\cup(\delta,\pi/2-\delta)$ which satisfy $\kappa^2 = \sin^2 (t_i+\phi)$, $i=1,2$. Moreover, for any $0<\lambda_1<1$ and $1<\lambda_2<1/\kappa$, there exists $\sigma>0$, which satisfy that $\chi_2=(t_1-\sigma,t_1+\sigma)\cup(t_2-\sigma,t_2+\sigma)\subset\chi_1$ and is only dependent on $\lambda_1,\lambda_2,\kappa$, such that
\be \label{assume1}
\lambda_1\kappa<|\sin (t+\phi)|<\lambda_2\kappa.
\ee
for any $t\in\chi_2$. We are now in a position to estimate $I_{21}$. Similarly, we split the path $L_1$ into $\chi_2$ and $L_1\bks \chi_2$, then we have the corresponding representation: $I_{21}=I_{\chi_2}+I_{L_1\bks{\chi_2}}$.

For $I_{\chi_2}$, we only analysis the integral on $\chi_{21}=(t_1-\sigma,t_1+\sigma)$ denoted by $I^1_{\chi_2}$, the procedure of the another is same. Without loss of generality, we assume that $\sin (t_1-\sigma+\phi)<\kappa<\sin (t_1+\sigma+\phi)$. It is easy to see that $\sin (t+\phi)$ is monotonic increasing in $\chi_{21}$. Let $\sin (t+\phi) = \kappa \sin \theta$ and the implicit mapping from $\theta$ to $t$ is denoted by $t(\theta)$ while the inverse mapping by $\theta(t)$, taking the interval $\chi_{21}$ onto $L_\theta : \theta_1\rightarrow\pi/2\rightarrow\pi/2-\i\theta_2$ where $\sin(t_1-\sigma+\phi)=\kappa\sin \theta_1,\sin(t_1+\sigma+\phi)=\kappa\sin(\pi/2-\i\theta_2)$. By substituting $t(\theta)$ into $I^1_{\chi_2}$, we have
\be
I^1_{\chi_2}=\int_{L_\theta}\frac{ h(\kappa\sin\theta,(1-\kappa^2\sin^2\theta)^{1/2},\kappa\cos\theta)}{(1-\kappa^2\sin^2\theta)^{1/2}} \ e^{\i
	k_s\rho(\cos(t(\theta)))} d\theta
\ee
Because of inequality \ref{assume1}, we assert that $h$ and $\pa h/\pa\theta$ are integrable on the path $L_\theta$. A simple computation show that
\ben
\frac{dt(\theta)}{d\theta}=\frac{\kappa\cos\theta}{\cos(t+\phi)} \ \
\frac{d^2 t(\theta)}{dt^2}=\frac{\kappa^2\cos^2\theta\sin(t+\phi)-\kappa\sin\theta\cos^2(t+\phi)}{\cos^3(t+\phi)}
\een
Then we can obtain
\ben
\frac{d\cos t}{d\theta}&=&\frac{-\kappa\sin t\cos\theta}{\cos(t+\phi)} \\
\frac{d^2\cos t}{d\theta^2}&=&\frac{d^2\cos t}{dt^2}(\frac{dt}{d\theta})^2+\frac{d\cos t}{dt}\frac{d^2t}{d\theta^2} \\
&=&\frac{-\kappa^2\cos^2\theta\cos t}{\cos^2(t+\phi)}+\frac{\kappa\sin\theta\cos^2(t+\phi)\sin t -\kappa^2\cos^2\theta\sin(t+\phi)\sin t}{\cos^3(t+\phi)} \\
&=&\frac{-\kappa^2\cos^2\theta\cos\phi+\kappa\sin\theta\cos^2(t+\phi)\sin t}{\cos^3(t+\phi)} \\
&=&\frac{(\sin^2(t+\phi)-\kappa^2)\cos\phi+\cos^2(t+\phi)\sin(t+\phi)\sin t}{\cos^3(t+\phi)}
\een
It is simple to see that $\theta=\pi/2$ is the only stationary point of $\cos(t(\theta))$ and we can obtain
\be
\Bigg|\frac{d^2\cos t}{d\theta^2}(\pi/2)\Bigg|=\frac{(1-\kappa^2)\kappa}{(1-\kappa^2)^{3/2}}|\sin t|>\frac{(1-\kappa^2)\kappa}{(1-\kappa^2)^{3/2}}\sin \delta
\ee
Therefore, we can choose appropriate $\lambda_1,\lambda_2$, only dependent on $\kappa$, such that $|\frac{d^2\cos t}{d\theta^2}|>\frac{(1-\kappa^2)\kappa}{(1-\kappa^2)^{3/2}}\sin \delta$ for any $\theta\in \theta(\chi_{21})$. Therefore, we can decompose $\theta(\chi_{21})$ into several intervals such that in each either $|\pa \cos(t(\theta))/\pa\theta|$ or $|\pa^2 \cos(t(\theta))/\pa\theta^2|$ has positive lower bound and $\pa \cos(t(\theta))/\pa\theta$ is monotonous. Since the amplitude function of integrand in $I^1_{\chi_2}$ and its derivative with respect to $\theta$ are both integrable on $L_\theta$, the estimation $|I^1_{\chi_2}|\leq C/(k_s\rho)^{1/2}$ can be obtained immediately by lemma \ref{van}. Then the estimate $|I_{L_1\bks{\chi_2}}|\leq C/(k_s\rho)^{1/2}$ also follows lemma \ref{van}. This completes the proof of the claim 1.

2. For the claim 2, since $\phi\neq\phi_\kappa$ we always can find some $\delta$ small enough such that $\sin^2(t+\phi)\neq \kappa^2$ for any $t\in(-\delta,\delta)$ .Thus, the proof is similar to the claim 1, here we omit the details.

3. To prove the claim 3, observe that 0 and $\phi'$ are the only two movable singular points of $f'(t)$ for on $L$ where $\sin^2{(\phi'+\phi_\kappa)}=\kappa^2$ and $\phi'\neq 0$.
However, we can not use stationary phase lemma directly because the fourth derivatives of amplitude function has singularity when $t=0$. Let $0<\delta<\pi/4$ such that $\sin (t+\phi_{\kappa})$ is monotonic as $t\in(-\delta,\delta)$ and $2\kappa/3\leq|\sin(\pm\delta+\phi_{\kappa})|\leq(1+2\kappa)/3$, then we have $I_1=I_{11}+I_{12}$,
$I_3=I_{31}+I_{32}$ similar to (\ref{I_split}). Using the same argument as in the proof of the claim 1, we can easily obtain
\be
|I_{12}|\leq C\frac{1}{k_s \rho} \ \ \ \  |I_{32}|\leq C\frac{1}{(k_s \rho)^{1/2}}
\ee
Obseve that $f(t)$ can be always represented as
\be
f(t)=\frac{g(t)}{\sin^2 (t+\phi_{\kappa})+\cos (t+\phi_{\kappa}) (\kappa^2-\sin^2 (t+\phi_{\kappa}))^{1/2}} \\
=\frac{g(t)(\sin^2(t+\phi_{\kappa})-\cos (t+\phi_{\kappa}) (\kappa^2-\sin^2(t+\phi_{\kappa}))^{1/2})}{(1+\kappa^2)\sin^2(t+\phi_{\kappa})-\kappa^2}  \\
=f_1(t)+f_2(t)(\sin^2 \phi_{\kappa} -\sin^2 (t+\phi_{\kappa}))^{1/2} \\
=f_1(t)+\i f_2(t)(\sin \phi_{\kappa}+\sin (t+\phi_{\kappa}))^{1/2}\cos^{1/2}(t/2+\phi_{\kappa}) (2\sin \frac{t}{2})^{1/2}  \\
=f_1(t)+g_1(t)(2\sin \frac{t}{2})^{1/2}
\ee
where $f_1,f_2,g_1\in C^\infty(-\delta,\delta)$. It follows that
\be \nn
I_{11}=\int_\R f_1(t)\cos t \chi_\delta(t)e^{\i k_s\rho\cos t} dt \\
+\int_\R g_1(t)\cos t (2\sin t/2)^{1/2} \chi_\delta(t)e^{\i k_s\rho\cos t} dt \\ \nn
:= I_{111}+I_{112}
\ee
\be \nn
I_{31}=\int_\R f_1'(t) \chi_\delta(t)e^{\i k_s\rho\cos t} dt \\
+\int_\R g_1'(t) (2\sin t/2)^{1/2} \chi_\delta(t)e^{\i k_s\rho\cos t} dt \\  \nn
+\int_\R 1/2g_1(t)\cos t (2\sin t/2)^{-1/2} \chi_\delta(t)e^{\i k_s\rho\cos t} \\ \nn
:= I_{311}+I_{312}+I_{313}
\ee
Subtitating $t(s)=2\arcsin s/2$ for t in $I_{112}$, $I_{312}$, and $I_{313}$ and by lemma (\ref{asym_frac}), we have
\be
|I_{112}|\leq C\frac{1}{(k_s \rho)^{3/4}},
|I_{312}|\leq C\frac{1}{(k_s \rho)^{3/4}},
|I_{313}|\leq C\frac{1}{(k_s \rho)^{1/4}}
\ee
By lemma (\ref{van}), we have
\be
|I_{311}|\leq C\frac{1}{(k_s \rho)^{1/2}}
\ee
Finally, the claim 3 is a direct cosequence of using stationary phase theorem for $I_{111}$. This completes the proof.
\finproof
\section{Some draft about Green Tensor Analysis}
Let substitute $\xi=k\sin\theta$ into integral and shift the variable, we have
\be
I(y)=\int_\R f(\xi)e^{\i \xi y_1+\mu(\xi) y_2} d\xi =\int_\R f(\xi)e^{\i \xi (y_1-z_1)+\mu(\xi) (y_2-z_2)}e^{\i \xi z_1 +\mu(\xi) z_2} d\xi \\
=k\int_{L} f(k\sin\theta)\cos\theta e^{\i k|y-z|\cos(\theta-\eta)}e^{\i|z|\cos(\theta-\phi)} d\theta\\
=k\int_{L_\phi} f(k\sin(\theta+\phi))\cos(\theta+\phi) e^{\i k|y-z|\cos(\theta+\phi-\eta)}e^{\i|z|\cos\theta} d\theta\\
=k\int_{L} f(k\sin(\theta+\phi))\cos(\theta+\phi) e^{\i k|y-z|\cos(\theta+\phi-\eta)}e^{\i|z|\cos\theta} d\theta
\ee
where $y_1,y_2>0,\sin\phi=\frac{z_1}{|z|},\cos\phi=\frac{z_2}{|z|},0<\phi<\pi/2$ and $\sin\eta=\frac{y_1-z_1}{|y-z|},\cos\eta=\frac{y_2-z_2}{|y-z|},0<\eta<\pi$. It is easy to see that $\phi+\eta<\pi$. Roughly, using stationary phase lemma, we obtain:
\be
I(y)=f(k\sin\phi)k\cos\phi e^{\i k|y-z|\cos(\phi-\eta)}(\frac{2\pi}{|z|})^{1/2}e^{\i|z|-\i\frac{\pi}{4}}(1+O(\frac{1}{|z|}))
\ee
\be
\cos(a+\i b)=\frac{e^b+e^{-b}}{2}\cos a+\i\frac{e^{-b}-e^b}{2}\sin a \\
\sin(a+\i b)=\frac{e^b+e^{-b}}{2}\sin a+\i\frac{e^{b}-e^{-b}}{2}\cos a
\ee
When $\theta\in(-a-\pi/2,-a-\pi/2+\i\infty)$, let $\theta=-a-\pi/2\i t$, where $t>0,0\leq
a\leq\phi$, then
\be\nn
-\Im(|z|\cos\theta+|y-z|\cos(\theta+\phi-\eta))\\
=|z|\sin(a+\pi/2)+|y-z|\sin(a+\pi/2-\phi+\eta) \\
=|z|\cos a+|y-z|\cos(a-\phi+\eta) \\
=|z|\cos a +\cos a|y-z|(\cos\phi\cos\eta+\sin\phi\sin\eta)\\
+\sin a |y-z|(\sin\phi\cos\eta-\cos\phi\sin\eta)\\
=|z|\cos a +\cos a((y_2-z_2)\cos\phi+(y_1-z_1)\sin\phi)\\
+\sin a((y_2-z_2)\sin\phi-(y_1-z_1)\cos\phi) \\
=y_1\sin(\phi-a)+y_2\cos(\phi-a)>0
\ee
Now, Using Cauchy Integral Theorem, we have
\be
I(y)=k\int_{L} f(k\sin(\theta+\phi))\cos(\theta+\phi) e^{\i k|y-z|\cos(\theta+\phi-\eta)}e^{\i|z|\cos\theta} d\theta
\ee
Let $L_1=(-\pi/2,-\pi/2+\i\infty)$ and $\theta=-\pi/2+\i t,t>0$, then
\be
I_1(y)=k\int_{L_1} f(k\sin(\theta+\phi))\cos(\theta+\phi) e^{\i k|y-z|\cos(\theta+\phi-\eta)}e^{\i|z|\cos\theta} d\theta \\
=
\ee
\be
I(y)=f(k\sin\phi)k\cos\phi e^{\i k|y-z|\cos(\phi-\eta)}(\frac{2\pi}{|z|})^{1/2}e^{\i|z|-\i\frac{\pi}{4}}\\
+\frac{kz_2}{|z|}O(\Bigg(\frac{1}{k|z|}\Bigg)^{3/4}+\frac{1}{k|y|})+\frac{kz_1}{|z|}O(\Bigg(\frac{1}{k|z|}\Bigg)^{5/4}+\Bigg(\frac{1}{k|y|}\Bigg)^2)
\ee
It is easy to see
\be
\int_{-d}^{d}\frac{k}{(k|x-z|)^\alpha}\frac{1}{(k|x-y|)^\beta} dx_1\leq C(\frac{1}{(kz_2)^{\alpha+\beta-1}}+\frac{1}{(ky_2)^{\alpha+\beta-1}})
\ee
where $z,y\in\R^2_+$, $x\in\Gamma_0$ and $\alpha+\beta>0$.
\be
e^{\i \mu  y_2+\i\xi(x_1-y_1)}=e^{\i\mu y_2 -\i y_2/\tan\phi}=e^{\i y_2(\mu-\xi/\tan\phi)}
\ee
Another method
\be
\int_{-\pi/2}^{\pi/2}f(k\sin(\theta+\psi))k\cos(\theta+\psi)e^{\i k|x-y|\cos\theta}\\
=\int_{-\pi/2}^{\pi/2}f(k\sin(\theta+\psi))k\cos(\theta+\psi)e^{\i k|x-y|\cos(\theta+\psi-\psi)} \\
=\int_{-\pi/2}^{\pi/2}f(k\sin(\theta+\psi))k\cos(\theta+\psi)e^{\i ky_2\cos(\theta+\psi)+\i k|x_1-y_1|\sin(\theta+\psi)} \\
=\int_{-\pi/2}^{\pi/2}f(k\sin(\theta+\psi))k\cos(\theta+\psi)\\
e^{\i k(y_2-z_2)\cos(\theta+\psi)+\i k(|x_1-y_1|-|x_1-z_1|)\sin(\theta+\psi)+\i k|z|\cos(\theta+\psi-\phi)}
\ee
\section{Finite Aperture Point Spread Function}
If $x\in \Gamma_0$ and $z,y\in \R^2_+$, by lemma (\ref{pr_dgreen}) we have
\be \hspace{-2.5cm} \nn
G(x,y)=\frac{\i k_s}{\mu\sqrt{2\pi}}\frac{1}{\delta(\xi)}\Bigg(
\begin{array}{cc}
	\mu_s\beta & \xi\beta \\
	2\xi\mu_s\mu_p & 2\xi^2\mu_p
\end{array}\Bigg)_{\xi=k_s\frac{x_1-y_1}{|x-y|}} \frac{y_2}{|x-y|}\frac{1}{(k_s|x-y|)^{1/2}}
e^{\i k_s|x-y|-\i\frac{\pi}{4}}\\ \hspace{-1cm}
+\frac{\i k_p}{\mu\sqrt{2\pi}}\frac{1}{\delta(\xi)}	
\Bigg(\begin{array}{cc}
	2\xi^2\mu_s & -2\xi\mu_s\mu_p\\
	-\xi\beta & \mu_p\beta
\end{array} \Bigg)_{\xi=k_p\frac{x_1-y_1}{|x-y|}} \frac{y_2}{|x-y|}\frac{1}{(k_p|x-y|)^{1/2}}
e^{\i k_p|x-y|-\i\frac{\pi}{4}} \\ \hspace{-1cm}\nn
+O(\frac{y_2}{|x-y|}\frac{1}{(k_s|x-y|)^{3/4}}+\frac{|x_1-y_1|}{|x-y|}\frac{1}{(k_s|x-y|)^{5/4}}) \\ \hspace{-1cm}\nn
:=\mathcal{G}_s(x,y)+\mathcal{G}_p(x,y)+O(\frac{y_2}{|x-y|}\frac{1}{(k_s|x-y|)^{3/4}}+\frac{|x_1-y_1|}{|x-y|}\frac{1}{(k_s|x-y|)^{5/4}})
\ee
\be
\hspace{-2.5cm} \nn
T_D(x,z)=\frac{k_s}{\sqrt{2\pi}}\frac{1}{\gamma(\xi)}\Bigg(
\begin{array}{cc}
	\mu_s\mu_p & \xi\mu_p \\
	\xi\mu_s & \xi^2
\end{array}\Bigg)_{\xi=k_s\frac{x_1-z_1}{|x-z|}} \frac{z_2}{|x-z|}\frac{1}{(k_s|x-z|)^{1/2}}
e^{\i k_s|x-z|-\i\frac{\pi}{4}}\\ \hspace{-1cm}
+\frac{ k_p}{\sqrt{2\pi}}\frac{1}{\gamma(\xi)}	
\Bigg(\begin{array}{cc}
	\xi^2 & -\xi\mu_p \\
	-\xi\mu_s & \mu_p\mu_s
\end{array} \Bigg)_{\xi=k_p\frac{x_1-z_1}{|x-z|}} \frac{z_2}{|x-z|}\frac{1}{(k_p|x-z|)^{1/2}}
e^{\i k_p|x-z|-\i\frac{\pi}{4}}\\ \hspace{-1cm} \nn
+O(\frac{k_s z_2}{|x-z|}\frac{1}{(k_s|x-z|)^{3/4}}+\frac{k_s|x_1-z_1|}{|x-z|}\frac{1}{(k_s|x-z|)^{5/4}}) \\ \hspace{-1cm}\nn
:=\mathcal{T}_s(x,z)+\mathcal{T}_p(x,z)+O(\frac{k_s z_2}{|x-z|}\frac{1}{(k_s|x-z|)^{3/4}}+\frac{k_s|x_1-z_1|}{|x-z|}\frac{1}{(k_s|x-z|)^{5/4}})
\ee
Now we consider the finite aperture point spread function $J_d(z,y)$:
\be
\int_{-d}^{d} (T_D(x_1,0;z_1,z_2))^T\overline{G(x_1,0;y_1,y_2)}dx_1
\ee
Recall following standard asymptotic expansion:
\be
|x-y|=|x-z|+\widehat{x-z}\cdot (z-y)+O(\frac{|y-z|^2}{|x-z|}) \\
|y|^{-\alpha}=|z|^{-\alpha}(1+\frac{|y|-|z|}{|z|})^{-\alpha}=|z|^{-\alpha}(1+O(\frac{|y-z|}{|z|})) \\
e^{\i t}=1+O(t) \\
|a^{1/2}-b^{1/2}|\leq C \sqrt{|a-b|}
\ee
where $x,y,z\in \R^2$, $t,a,b\in\R $ and $\alpha>0$.
\begin{lem}
For any $z,y\in \R^2_+$, $J_d(z,y)=F(z,y)+O((1+\frac{|y-z|}{z_2})(\frac{1}{k_s z_2})^{1/4}+\frac{(k_s|y-z|)^2}{k_s z_2}+(\frac{|y-z|}{z_2})^{1/2})$, where
\be \hspace{-2.2cm}
F(z,y) =-\frac{\i}{2\pi\mu}\int_{\theta^d_1}^{\theta^d_2}f_s(\theta)
\Bigg(
\begin{array}{cc}
  \sin^2\theta & \sin\theta\cos\theta  \\
  \sin\theta\cos\theta & \cos^2\theta
\end{array}\Bigg)
e^{\i k_s(z_1-y_1)\cos\theta+\i k_s(z_2-y_2)\sin\theta} d\theta \\ \hspace{-1.2cm}
-\frac{\i}{2\pi\mu} \int_{\theta^d_1}^{\theta^d_2}f_p(\theta)
\Bigg(
\begin{array}{cc}
  \cos^2\theta & -\sin\theta\cos\theta  \\
  -\sin\theta\cos\theta & \sin^2\theta
\end{array}\Bigg)
e^{\i k_p(z_1-y_1)\cos\theta+\i k_p(z_2-y_2)\sin\theta}d\theta
\ee
and
\ben\hspace{-3cm}
f_s(\theta)=\frac{\sin\theta((\kappa^2-\cos^2\theta)^{1/2}(1-2\cos^2\theta)+2\overline{(\kappa^2-\cos^2\theta)^{1/2}}\cos^2\theta)}
{(\cos^2\theta+\sin\theta(\kappa^2-\cos^2\theta)^{1/2})\overline{((1-2\cos^2\theta)^2+4\cos^2\theta\sin\theta(\kappa^2-\cos\theta)^{1/2}})} \\\hspace{-3cm}
f_p(\theta)=\frac{\sin\theta(1/\kappa^2-\cos^2\theta)^{1/2}}
{(\cos^2\theta+\sin\theta(1/\kappa^2-\cos^2\theta)^{1/2})((1/\kappa^2-2\cos^2\theta)^2+4\cos^2\theta\sin\theta(1/\kappa^2-\cos\theta)^{1/2})}
\een
where $0<\theta^d_1<\pi/2<\theta^d_2<\pi$ and $z_2=(d+z_1)\tan \theta^d_1=(z_1-d)\tan \theta^d_2$.
\end{lem}
\debproof
\ben
\frac{y_2}{|x-y|}\frac{1}{(k_s|x-y|)^{3/4}}+\frac{|x_1-y_1|}{|x-y|}\frac{1}{(k_s|x-y|)^{5/4}} \\ \hspace{-0.5cm}
=(\frac{ z_2}{|x-z|}\frac{1}{(k_s|x-z|)^{3/4}}+\frac{|x_1-z_1|}{|x-z|}\frac{1}{(k_s|x-z|)^{5/4}})(1+O(\frac{|y-z|}{|x-z|}))
\een
\ben
|\mu_\i(k_j\frac{x_1-y_1}{|x-y|})-\mu_\i(k_j\frac{x_1-z_1}{|x-z|})| \\  \leq
C k_j \sqrt{\Big|\frac{x_1-y_1}{|x-y|}-\frac{x_1-z_1}{|x-z|}\Big|} \leq C k_j \Big(\frac{|y-z|}{|x-z|}\Big)^{1/2}
\een
where $i,j=s,p$. By above, we can obtain
\be \label{G_sprin}\hspace{-2cm}
\mathcal{G}_s(x,y)=\mathcal{G}_s(x,z)e^{\i k_s\widehat{x-z}\cdot (z-y)}+O(\frac{(k_s|y-z|)^2}{(k_s|x-z|)^{3/2}})+O(\frac{(k_s|y-z|)^{1/2}}{k_s|x-z|}) \\ \hspace{-2cm}\label{G_pprin}
\mathcal{G}_p(x,y)=\mathcal{G}_p(x,z)e^{\i k_p\widehat{x-z}\cdot (z-y)}+O(\frac{(k_p|y-z|)^2}{(k_p|x-z|)^{3/2}})+O(\frac{(k_p|y-z|)^{1/2}}{k_p|x-z|})
\ee
For $l>1$, a simple computation show that
\be\label{int_d} \hspace{-1cm}
\int_{-d}^{d}\frac{k_s }{(k_s|x-z|)^l}dx_1=\frac{1}{(k_s z_2)^{l-1}}\int_{\frac{-d-z_1}{z_2}}^{\frac{d-z_1}{z_2}}\frac{1}{(1+t^2)^{l/2}} dt\leq C\frac{1}{(k_s z_2)^{l-1}}
\ee
Let
\be
\mathcal{G}_\alpha(x,y)=\frac{\i}{\sqrt{2\pi}\mu}g_\alpha(\frac{x_1-y_1}{|x-y|},\kappa)\frac{1}{(k_\alpha|x-y|)^{1/2}}e^{\i k_\alpha|x-y|-\i\frac{\pi}{4}} \\
\mathcal{T}_\alpha(x,y)=\frac{k_\alpha}{\sqrt{2\pi}}
 t_\alpha(\frac{x_1-z_1}{|x-z|},\kappa)\frac{1}{(k_s|x-z|)^{1/2}}
e^{\i k_\alpha|x-z|-\i\frac{\pi}{4}}
\ee
where $\alpha=s,p$.
Now, by substituting (\ref{G_sprin}-\ref{G_pprin}) into $J_d(z,y)$ and using inequality (\ref{int_d}), we have
\be\nn\hspace{-1.5cm}\nn
J_d(z,y)=\frac{-\i}{2\pi\mu}\int_{-d}^{d}t_s(\frac{x_1-z_1}{|x-z|},\kappa)^T\overline{g_s(\frac{x_1-z_1}{|x-z|},\kappa)}\frac{e^{\i k_s\widehat{x-z}\cdot (y-z)}}{|x-z|}\\
+t_p(\frac{x_1-z_1}{|x-z|},\kappa)^T\overline{g_p(\frac{x_1-z_1}{|x-z|},\kappa)}\frac{e^{\i k_p\widehat{x-z}\cdot (y-z)}}{|x-z|}dx_1\\
-\frac{\i}{2\pi\mu}\int_{-d}^{d}t_p(\frac{x_1-z_1}{|x-z|},\kappa)^T\overline{g_s(\frac{x_1-z_1}{|x-z|},\kappa)}\frac{e^{\i k_s\widehat{x-z}\cdot (y-z)}}{|x-z|}\\
+t_s(\frac{x_1-z_1}{|x-z|},\kappa)^T\overline{g_p(\frac{x_1-z_1}{|x-z|},\kappa)}\frac{e^{\i k_p\widehat{x-z}\cdot (y-z)}}{|x-z|}dx_1\\
+O((1+\frac{|y-z|}{z_2})(\frac{1}{k_s z_2})^{1/4}+\frac{(k_s|y-z|)^2}{k_s z_2}+(\frac{|y-z|}{z_2})^{1/2})\\
:=F(z,y)+R(z,y)\\
+O((1+\frac{|y-z|}{z_2})(\frac{1}{k_s z_2})^{1/4}+\frac{(k_s|y-z|)^2}{k_s z_2}+(\frac{|y-z|}{z_2})^{1/2})
\ee
We denote $\widehat{x-z}=x-z/|x-z|=(cos(\phi+\pi),\sin(\phi+\pi))$, then taking the substitution $x_1=z_1-z_2\cot\phi$, we obtain
\be
F(z,y)&=&\frac{-\i}{2\pi\mu} \int_{\theta^d_1}^{\theta^d_2}A_s(\phi,\kappa)e^{\i k_s(z_1-y_1)\cos\phi+\i k_s(z_2-y_2)\sin\phi} \\
&+&\frac{-\i}{2\pi\mu} \int_{\theta^d_1}^{\theta^d_2}A_p(\phi,\kappa)e^{\i k_p(z_1-y_1)\cos\phi+\i k_p(z_2-y_2)\sin\phi}
\ee
\be
R(z,y)&=&\frac{-\i}{2\pi\mu} \int_{\theta^d_1}^{\theta^d_2}B_s(\phi,\kappa)e^{\i k_s(z_1-y_1)\cos\phi+\i k_s(z_2-y_2)\sin\phi+(k_p-k_s)|x-z|} \\
&+&\frac{-\i}{2\pi\mu} \int_{\theta^d_1}^{\theta^d_2}B_p(\phi,\kappa)e^{\i k_p(z_1-y_1)\cos\phi+\i k_p(z_2-y_2)\sin\phi+(k_s-k_p)|x-z|}
\ee
It is easy to see that $|R(z,y)|\leq C\frac{|z-y|}{z_2}$.
\finproof

\section*{References}
\bibliography{eee}
\end{document}
